\documentclass[12pt]{article}
\usepackage{csvsimple}
\usepackage{fullpage}
\usepackage{hyperref}
\begin{document}
\title{Using co-SAT solvers to find lower bounds on detecting triangles}
\author{Josh Burdick}
\maketitle

\begin{abstract}
Detecting whether an undirected
graph contains a triangle (a.k.a. 3-clique, or $K_3$)
is a tiny case of the NP-complete
CLIQUE problem. We would therefore expect it to be both hard to answer
quickly, and also difficult
to {\em prove} hard to answer quickly. Here, we encode the problem of
finding a NAND gate circuit which detects triangles, as a CNF formula
(in DIMACS-CNF format,
suitable as input for SAT or co-SAT solvers). This provides a
possible source of either lower bounds, or difficult co-SAT instances
(although it's not clear which).
\end{abstract}

The problem of
detecting triangles has been studied, partly because it is a tiny special
case of detecting cliques, which is NP-complete.
In circuit terms,
triangles can be detected using ``brute force'', using ${n \choose 3}$ AND gates
(one per possible triangle), and an OR gate to combine the outputs of those;
or alternatively with
${n \choose 3} + 1$
NAND gates.

The triangle detection problem
can also essentially be solved by multiplying two Boolean matrices
together \cite{itai_finding_1977}. Successively faster algorithms allow doing this in
$O(n^{2.8074})$ \cite{strassen_gaussian_1969} or
$O(n^{2.3727})$ \cite{williams_multiplying_2012} operations. The big-$O$ notation
hides large constants, and the ``operations''
counted in analysis of those methods are multiplications, which can involve large
numbers.
However, to multiply $n$-by-$n$ Boolean matrices, we can use addition and multiplication
mod $n+1$ (FIXME cite CLR), avoiding computing large numbers.
Thus, such matrix multiplication algorithms allow triangle detection
which asymptotically use fewer than $\Omega(n^3)$ NAND gates.
\footnote{In a previous version of this, I didn't realize this, and so thought
that the ${n \choose 3} + 1$ circuit might be optimal; I now realize that it isn't.}
Nonetheless, finding bounds on the number of NAND gates needed to detect
triangles in small graphs may be interesting.

Previously, I wrote up an attempt at a proof that
finding triangles in 6-input graphs requires
${6 \choose 3} + 1 = 21$ NAND gates \cite{burdick_triangles_6_vertex}.
(The proof has not been thoroughly checked, or published.)
I tried pushing the same methods to get a proof for the
$n=7$ case, but they didn't seem to be going much higher
than ``slightly more than the number of inputs.'' 
(I don't think this is very surprising.)

Therefore, I decided to try applying brute force, using solvers for propositional
logic. Development of such solvers has been accelerated by their use in logic
circuit design.
I first tried encoding the problem
as a quantified Boolean formula, and, using a BDD solver,
looking for a circuit which finds
triangles (or, more importantly, ruling out
the existence of such a circuit). Since that was using a large amount of memory,
I then tried using a SAT (or co-SAT) solver to look
for a circuit (or rule out the possibility of a circuit)
which only {\em approximately} finds triangles.

\subsection{Motivation}

Why might bounds on circuits for triangle-finding in tiny graphs be useful? Since we're
only looking at tiny input graphs (and circuits), such bounds have
limited relevance to the $P=NP$ question.

It seems natural to guess that ``you need at least one NAND gate per triangle'',
and try to prove that.
(This conjecture seems related to the Strong Exponential-Time Hypothesis, or SETH \cite{impagliazzo_complexity_1999}.)
(However, as noted earlier, using fast multiplication methods, you don't need
at least one NAND gate per triangle.)

The encoded formulas may also be useful as test problems for SAT solvers.
Showing that a formula is not satisfiable is an important problem, and
benchmarks and competitions for solvers of such
problems often require a source of
test problems known to be unsatisfiable, and/or difficult to solve \cite{toughsat}.
 Assuming $P \ne NP$,
a formula corresponding to the existence of a small circuit for an NP-complete
problem should be unsatisfiable. (To reduce the problem size, we relax the requirement
that the circuit must work on all possible inputs; this means that the formula we
write may be satisfiable, even if no circuit with that many gates {\em exactly}
finds triangles.)

Note that this is somewhat win-win: either the generated formulas are easy to
prove unsatisfiable (in which case we get better lower bounds on triangle finding),
or they're hard to prove unsatisfiable (in which case we have more probably-unsatisfiable
problems for testing co-SAT provers; admittedly this may be less of a win.)

\section{Encoding as a quantified Boolean formula}

One convenience of looking for bounds on NAND gate circuits
(as opposed to, say, circuits with AND, OR, and NOT gates,
or arbitrary Boolean functions) is that the gates are all
the same kind. The circuit is defined only by its
connectivity graph.

This means that we can encode a circuit
using a Boolean variable for each wire which might be
present. Specifically, we can encode an $m$-input circuit
made of $g$ NAND gates using $mg$ bits (one per input-gate
wire), plus $g \choose 2$ bits (one per gate-gate wire).
We also have one variable per input wire (edge), and can code up
a formula for ``has a triangle.''
To code up the formula ``there is a circuit which works'', we
have to universally quantify over the inputs (because the
circuit has to work on all inputs). This causes a
blowup in the formula size.

When I ran this, with four vertices and five gates, it
fairly quickly found a working circuit. 
\footnote{I didn't actually print out the solution for $n=4$, with
five gates, so it could be incorrect. Given the small numbers
involved, I'm not planning on bothering with this further.}
For five vertices, it ruled out a circuit with five gates.
I tried larger numbers of gates (six or seven, I don't recall
which, but definitely less than the known lower bound of eleven
gates), and it just allocated memory until it crashed
my machine (a Thinkpad X200s with 8gb memory.)

\subsection{Forcing wires to be sorted}

I also tried adding a constraint that the rows of
the connectivity matrix are in lexicographic order.
We can do this without loss of generality. Proof sketch:
pick an arbitrary order for the input wires. For all gates
which only have front-level inputs, write their inputs as
a bit-vector, find the lexicographically first, and call it
$g_1$. Then find $g_2$ by only considering front-level inputs,
and (possible) wires from $g_1$, treating it as the most significant bit.
Repeat until you've added all the gates.
This defines a unique ordering of gates, such that the adjacency
matrix of all wires is lexicographically sorted.

This may have helped a bit, but not much (data not shown; I
look at the effect of this more carefully below.)

\section{Approximate triangle-finding circuits}

In the previous attempt, we quantified over all the possible inputs. This meant that the notion of
whether there was a triangle was part of the formula. As in Razborov's monotone circuit bound \cite{Razborov85lowerbounds}, we
can also approximately specify how the circuit should work.
MAX-CLIQUE, at least, is
hard to approximate, unless $P=NP$ \cite{hastad_clique_1999}, so we may not have to
specify every possible input graph to force the circuit to have many gates.
Since the resulting formula
is just Boolean (without quantifiers), I used a SAT solver \cite{een_extensible_2003}
rather than a BDD solver.

This approach is reminiscent of machine learning: we are asking for a ``simple'' circuit
which explains some examples. At first,
I used the $n \choose 3$ cases which have one triangle (and zeros elsewhere) as
positive examples. Following Razborov's monotone circuit bound \cite{Razborov85lowerbounds}, 
I used complete bipartite graphs as negative examples.

This small number of examples was enough to show that, with four vertices,
at least five gates were needed. However, with five vertices, there were
circuits which distinguished those test cases, but had fewer than ${5 \choose 2} + 1 = 11$ gates.
Therefore, I added ``test cases'' which were random graphs,
and added a constraint that the circuit would output 0 or 1 (according
to whether the graph contained a triangle.)

\section{Results}

For a small set of tiny problems, the following table shows whether the problem was satisfiable
(and there is a circuit with that many gates that is correct, at least given the number of
sample inputs given). Also shown is the runtime, on a
lightly-loaded X200s laptop. (Where ``satisfiable?'' is -, I stopped the program by hand.)

\csvautotabular{runtimes.csv}

\section{Conclusions}

This approach didn't improve the lower bounds. However, it was
less memory-intensive than using BDDs. It may be a useful source
of ``likely to be unsatisfiable'' SAT instances.
Forcing the wires' adjacency matrix to be sorted didn't help much.

\subsection{Future work}

Rather than using random graphs as training data, it might be possible to choose those
more systematically. Possibly graphs with multiple triangles would be 
better positive examples.
Or, perhaps two-edge paths (which are triangles with an edge removed) would be better negative
examples.

Boiling the problem down to propositional logic presumably loses a lot of the inherent
symmetry in the problem. However, fully automated SAT solvers (that is, zero-order
theorem provers) are arguably more developed
than fully automated first- or higher-order theorem provers.

\bibliography{references}
\bibliographystyle{unsrt}

\end{document}

